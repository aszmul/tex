% This is the externalization variant of the tikztension, i.e., the one that 
% should be used to send to journals etc. It assumes that all graphics have been 
% generated in an appropriate format (pdf, ps, or eps) already. This is how you 
% can get rid of the tikz-dependence for publications.
\usepackage{graphics}

% single escape
%
% escapes a single character
\def\sesc#1{%
\if#1.-%   . -> -
\else%
\if#1=-%   = -> -
\else%
\if#1,-%   , -> -
\else%
\if#1/-%   / -> -
\else#1%
\fi%
\fi%
\fi%
\fi%
}

% denotes the end of the sequence to escape
\def\escnull{}

% escapes a sequence that ends with \escnull followed by \escend
\def\esc#1#2\escend{%
\ifx#1\escnull%
\else%
\sesc#1\esc#2\escend%
\fi%
}

% Escapes a string using the replacements in \sesc. Introduces no non-expandable 
% macros, i.e., the result can safely be used for filenames.
\def\escape#1{\esc#1\escnull\escend}

% #1 is the base name of the tikz file
% #2 is a string of PGF keys that are passed to \begin{tikzpicture}. It may 
% contain the special key /tikztension/define that defines empty macros. These 
% can be used as flags in the tikzimage via \ifdefined.
\newcommand{\includetikz}[2][]{%
  \def\graphicname{#2-\escape{#1}}%
  % include externalized graphic
  \includegraphics{\graphicname}%
}%
