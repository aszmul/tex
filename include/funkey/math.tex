% This file provides definitions of mathematical operators, symbols, and theorem 
% environments.


% defining equality
\newcommand{\define}{\stackrel{\text{\tiny def}}{=}}

% argmin, argmax
\newcommand{\argmin}[1]{\mathop{\arg\min}_{#1}\hspace{0.5em}}
\newcommand{\mmin}[1]{\mathop{\min}_{#1}\hspace{0.5em}}
\newcommand{\argmax}[1]{\mathop{\arg\max}_{#1}\hspace{0.5em}}
\newcommand{\mmax}[1]{\mathop{\max}_{#1}\hspace{0.5em}}

% const
\newcommand\const{\text{const}}

% transpose and inverse
\newcommand{\trans}[0]{{\sf T}}
\newcommand\inv{{-1}}

% annotations in equations
\newcommand{\annotation}[1]{& \text{{\small{#1}}}}

% function names
\newcommand\gauss{\mathcal{N}}

% ILP
%
% Usage:
%
% \beginilp
%   \objective{\
\def\beginilp{%
  \makeatletter
  \global\let\ilp@firstconstraint 1
  \makeatother
  % {min,s.t.}[spacer]{objective,lhs}[rel][rhs][qualifier]
  \begin{IEEEeqnarray}{r?rCl"l}
}
\def\objective#1#2{%
  #1
  &
  \IEEEeqnarraymulticol{3}{l}{#2}
}
\def\constraint#1#2#3#4{%
  \\
  \makeatletter
  \if\ilp@firstconstraint 1 \text{s.t.} \fi
  \global\let\ilp@firstconstraint 0
  \makeatother
  &
  #1
  &
  #2
  &
  #3
  &
  #4
  \nonumber
}
\def\endilp{%
  \end{IEEEeqnarray}
}
