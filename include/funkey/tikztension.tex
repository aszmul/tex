% This file provides all the magic for our TikZ-externalisation. Whenever you 
% include your TikZ picture example.tikz.tex using \includetikz{example}, a pdf 
% file with the image only will be generated and included instead. This prevents 
% TikZ pictures from being re-typeset every time you compile your document.  
% Also, it facilitates interaction with non-TikZ submission systems, since all 
% pictures are present as pdfs already and you don't need TikZ anymore to 
% compile the document (try 'make pdfexport').

\usepackage{tikz}
\usepackage{pgfplots}
\usepackage{todonotes}

\newwrite\depfile
\immediate\openout\depfile=tikz.dep

\ifdefined\enabletikztension
  \pgfrealjobname{everything}
\else
  \pgfrealjobname{\jobname}
\fi

% single escape
%
% escapes a single character
\def\sesc#1{%
\if#1.-%   . -> -
\else%
\if#1=-%   = -> -
\else%
\if#1,-%   , -> -
\else%
\if#1/-%   / -> -
\else#1%
\fi%
\fi%
\fi%
\fi%
}

% denotes the end of the sequence to escape
\def\escnull{}

% escapes a sequence that ends with \escnull followed by \escend
\def\esc#1#2\escend{%
\ifx#1\escnull%
\else%
\sesc#1\esc#2\escend%
\fi%
}

% Escapes a string using the replacements in \sesc. Introduces no non-expandable 
% macros, i.e., the result can safely be used for filenames.
\def\escape#1{\esc#1\escnull\escend}

% Creates the basename of the pdf file and writes the dependency to the original 
% tikz file to tikz.dep.
\def\writedep#1#2{%
  \def\basename{#2-\escape{#1}}%
  \immediate\write\depfile{\basename.pdf: #2.tikz.tex}%
}

% make the @ a letter
\catcode`\@=11

% a key that defines an empty macro with the name of the value of the key
\pgfkeys{/tikztension/define/.code=%
  \def\tikztension@defined{#1}%
  \expandafter\def\csname #1\endcsname{}%
  %(defined #1)
}

% introduce a new family of keys
\pgfkeys{
  /tikztension/.is family,
  /tikztension/define/.belongs to family=/tikztension}
\pgfkeys{/pgf/key filters/active families/.install key filter}

% #1 is the base name of the tikz file
% #2 is a string of PGF keys that are passed to \begin{tikzpicture}. It may 
% contain the special key /tikztension/define that defines empty macros. These 
% can be used as flags in the tikzimage via \ifdefined.
\newcommand{\includetikz}[2][]{%
  \writedep{#1}{#2}%
  \def\graphicname{#2-\escape{#1}}%
  % preparation
  \expandafter\edef\csname %
    tikztension@atcode\endcsname{\the\catcode`\@}%
  \catcode`@=11%
  \def\remainingoptions{}%
  \def\tikztension@defined{}%
  % collect all filtered options in /remainingoptions
  \pgfkeys{/pgf/key filter handlers/%
           append filtered to/%
           .install key filter handler=
           \remainingoptions}
  % The following is a bit tricky:
  % First, /tikztension is made active, such that only options from this family 
  % are getting processed. Then we expand the search path to /tikz, such that we 
  % do not get "don't know a key with name..." error whenever we encounter a 
  % tikz option we do not care about anyway (it is not in the active family). In 
  % the following call to pgfkeysfiltered we first set the current path to 
  % /tikz, then to /tikztension, and then we process the options. In a normal 
  % call to /pgfkeys the first option (/tikz) would have no effect. In our case, 
  % however, it will be filtered out and prepended to the /remainingoptions.  
  % This way the /tikz options in #1 do not have to be prefixed by /tikz!
  \pgfkeys{/tikztension/.activate family}%
  \pgfkeys{/tikztension/.search also={/tikz}}%
  \pgfkeysfiltered{tikz,/tikztension/.cd,#1}%
  % include tikz graphic or pdf file
  \expandafter\beginpgfgraphicnamed{\graphicname}%
    \IfFileExists{#2.tikz.tex}%
    {%
      \begin{tikzpicture}%
        \pgfkeysalsofrom{\remainingoptions}%
        \node[transform shape] {\input{#2.tikz.tex}};%
      \end{tikzpicture}%
    }%
    {image file does not exist (yet)}%
  \endpgfgraphicnamed%
  % unset tikztension option
  \expandafter\let\csname\tikztension@defined\endcsname=\@undefined
  % change back to original catcode of @
  \catcode`\@=\csname tikztension@atcode\endcsname
}%
