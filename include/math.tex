% This file provides definitions of mathematical operators, symbols, and theorem 
% environments.


% defining equality
\newcommand{\define}{\stackrel{\text{\tiny def}}{=}}

% argmin, argmax
\newcommand{\argmin}[1]{\mathop{\arg\min}_{#1}\hspace{0.5em}}
\newcommand{\mmin}[1]{\mathop{\min}_{#1}\hspace{0.5em}}
\newcommand{\argmax}[1]{\mathop{\arg\max}_{#1}\hspace{0.5em}}
\newcommand{\mmax}[1]{\mathop{\max}_{#1}\hspace{0.5em}}

% const
\newcommand\const{\text{const}}

% transpose and inverse
\newcommand{\trans}[0]{{\sf T}}
\newcommand\inv{{-1}}

% annotations in equations
\newcommand{\annotation}[1]{& \text{{\small{#1}}}}

% function names
\newcommand\gauss{\mathcal{N}}

% theorem etc. environments

\usepackage{framed}
\usepackage[framed,thmmarks]{ntheorem}

% general theorem style
\theoremstyle{plain}
\theorembodyfont{}
\def\theoremframecommand{\fcolorbox{black}{gray!20!white}}

\newshadedtheorem{theorem}{Theorem}[section]

\newshadedtheorem{lemma}[theorem]{Lemma}

\newshadedtheorem{proposition}[theorem]{Proposition}

\newshadedtheorem{corollary}[theorem]{Corollary}

\newshadedtheorem{definition}{Definition}[section]

{
\theoremstyle{nonumberplain}
\theoremsymbol{\rule{1ex}{1ex}}
\theoremseparator{:}
\newtheorem{proof}{Proof}[section]
}
